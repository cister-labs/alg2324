\documentclass[aspectratio=169]{beamer}
\usepackage{etex} % fixes new-dimension error
\usepackage{lmodern}
\usepackage[T1]{fontenc}

\usepackage{graphicx,amsmath}
\usepackage{stmaryrd} % cf. interleave
\usepackage{booktabs}
\usepackage{amscd}
\usepackage{multicol}
\usepackage[absolute,overlay]{textpos}
\usepackage{alltt}
\usepackage{proof}
\usepackage{subcaption}
\usepackage{tikz}
\usepackage{tikz-cd}
\usepackage[new]{old-arrows}
\usepackage[all]{xy}
\usepackage{pgfplots}
\usepackage{textcomp}

\usepackage{transparent}
\usepackage{xspace}
\usepackage{listings}
\usepackage{pdfpages}
\usepackage{relsize}

%%%%%%%%%%%%% Macros
\newcommand{\Ban}{\catfont{Ban}}
\newcommand{\Met}{\catfont{Met}}
\newcommand{\Shuff}{\mathrm{Sf}}
\newcommand{\Cats}{\catfont{Cat}}
\newcommand{\VCat}{\mathcal{V}\text{-}\Cats}
\newcommand{\VCatSy}{\mathcal{V}\text{-}\Cats_{\mathsf{sym}}}
\newcommand{\VCatSe}{\mathcal{V}\text{-}\Cats_{\mathsf{sep}}}
\newcommand{\VCatSS}{\mathcal{V}\text{-}\Cats_{\mathsf{sym,sep}}}
%%%% Categories
\newcommand{\catfont}[1]{\mathsf{#1}}
\newcommand{\cop}{\catfont{op}}
\newcommand{\Law}{\catfont{Law}}
\newcommand{\catV}{\catfont{V}}
\newcommand{\catX}{\catfont{X}}
\newcommand{\catC}{\catfont{C}}
\newcommand{\catD}{\catfont{D}}
\newcommand{\catA}{\catfont{A}}
\newcommand{\catB}{\catfont{B}}
\newcommand{\catI}{\catfont{I}}
\newcommand{\Set}{\catfont{Set}}
\newcommand{\Top}{\catfont{Top}}
\newcommand{\Pos}{\catfont{Pos}}
\newcommand{\Inj}{\catfont{Inj}}
\newcommand{\Det}{\catfont{RMhat}}
\newcommand{\CoAlg}[1]{\catfont{CoAlg}\left (#1 \right )}
\newcommand{\Mon}{\catfont{Mon}}
\newcommand{\Mnd}{\catfont{Mnd}(\catC)}
\newcommand{\SMnd}{\catfont{Mnd}(\Set)}
\newcommand{\CLat}{\catfont{CLat}}
\newcommand{\Stone}{\catfont{Stone}}
\newcommand{\Spectral}{\catfont{Spectral}}
\newcommand{\CompHaus}{\catfont{CompHaus}}
\newcommand{\Subs}[2]{\catfont{Sub}_{}}
\newcommand{\Cone}{\catfont{Cone}}
\newcommand{\StComp}{\catfont{StablyComp}}
\newcommand{\PosC}{\catfont{PosComp}}
\newcommand{\Haus}{\catfont{Haus}}
\newcommand{\Meas}{\catfont{Meas}}
\newcommand{\Ord}{\catfont{Ord}}
\newcommand{\EndoC}{[\catC,\catC]}
%% General functors
\newcommand{\funfont}[1]{#1}
\newcommand{\funF}{\funfont{F}}
\newcommand{\funU}{\funfont{U}}
\newcommand{\funG}{\funfont{G}}
\newcommand{\funT}{\funfont{T}}
\newcommand{\funI}{\funfont{I}}
%% Particular kinds of functors
\newcommand{\sfunfont}[1]{\mathrm{#1}}
\newcommand{\Pow}{\sfunfont{P}}
\newcommand{\Dist}{\sfunfont{D}}
\newcommand{\Maybe}{\sfunfont{M}}
\newcommand{\List}{\sfunfont{L}}
\newcommand{\UForg}{\sfunfont{U}}
\newcommand{\Forg}[1]{\sfunfont{U}_{#1}}
\newcommand{\Id}{\sfunfont{Id}}
\newcommand{\Vie}{\sfunfont{V}}
\newcommand{\Disc}{\funfont{D}}
\newcommand{\Weight}{\sfunfont{W}}
\newcommand{\homf}{\sfunfont{hom}}
\newcommand{\Yoneda}{\sfunfont{Y}}
%% Diagram functors
\newcommand{\Diag}{\mathscr{D}}
\newcommand{\KDiag}{\mathscr{K}}
\newcommand{\LDiag}{\mathscr{L}}
%% Monads
\newcommand{\monadfont}[1]{\mathbb{#1}}
\newcommand{\monadT}{\monadfont{T}}
\newcommand{\monadS}{\monadfont{S}}
\newcommand{\monadU}{\monadfont{U}}
\newcommand{\monadH}{\monadfont{H}}
\newcommand{\str}{\mathrm{str}}
%% Adjunctions
\newcommand\adjunct[2]{\xymatrix@=8ex{\ar@{}[r]|{\top}\ar@<1mm>@/^2mm/[r]^{{#2}}
& \ar@<1mm>@/^2mm/[l]^{{#1}}}}
\newcommand\adjunctop[2]{\xymatrix@=8ex{\ar@{}[r]|{\bot}\ar@<1mm>@/^2mm/[r]^{{#2}}
& \ar@<1mm>@/^2mm/[l]^{{#1}}}}
%% Retractions
\newcommand\retract[2]{\xymatrix@=8ex{\ar@{}[r]|{}\ar@<1mm>@/^2mm/@{^{(}->}[r]^{{#2}}
& \ar@<1mm>@/^2mm/@{->>}[l]^{{#1}}}}
%% Limits
\newcommand{\pv}[2]{\langle #1, #2 \rangle}
\newcommand{\limt}{\mathrm{lim}}
\newcommand{\pullbackcorner}[1][dr]{\save*!/#1+1.2pc/#1:(1,-1)@^{|-}\restore}
\newcommand{\pushoutcorner}[1][dr]{\save*!/#1-1.2pc/#1:(-1,1)@^{|-}\restore}
%% Colimits
\newcommand{\colim}{\mathrm{colim}}
\newcommand{\inl}{\mathrm{inl}}
\newcommand{\inr}{\mathrm{inr}}
%% Distributive categories
\newcommand{\distr}{\mathrm{dist}}
\newcommand{\undistr}{\mathrm{undist}}
%% Closedness
\newcommand{\curry}[1]{\mathrm{curry}{#1}}
\newcommand{\app}{\mathrm{app}}
%% Misc. operations
\newcommand{\const}[1]{\underline{#1}}
\newcommand{\comp}{\cdot}
\newcommand{\id}{\mathrm{id}}
%% Factorisations
\newcommand{\EClass}{E}
\newcommand{\MClass}{M}
\newcommand{\MConeClass}{\mathcal{M}}
%%%%%%%%%%%%%%%% End of Categorical Stuff

%%%% Misc
%% Operations
\newcommand{\blank}{\, - \,}
\newcommand{\sem}[1]{\llbracket #1 \rrbracket}
\newcommand{\closure}[1]{\overline{#1}}
\DeclareMathOperator{\img}{\mathrm{im}}
\DeclareMathOperator{\dom}{\mathrm{dom}}
\DeclareMathOperator{\codom}{\mathrm{codom}}
%% Sets of numbers
\newcommand{\N}{\mathbb{N}}
\newcommand{\Z}{\mathbb{Z}}
\newcommand{\Nats}{\mathbb{N}}
\newcommand{\Reals}{\mathbb{R}}
\newcommand{\Rz}{\Reals_{\geq 0}}
\newcommand{\Complex}{\mathbb{C}}
%% Writing
\newcommand{\cf}{\emph{cf.}}
\newcommand{\ie}{\emph{i.e.}}
\newcommand{\eg}{\emph{e.g.}}
\newcommand{\df}[1]{\emph{\textbf{#1}}}
%%%%%%%%%%%%%%%% End of Misc

%%%% Programming Stuff
%% Types
\newcommand{\typefont}[1]{\mathbb{#1}}
\newcommand{\typeOne}{1}
\newcommand{\typeTwo}{2}
\newcommand{\typeA}{\typefont{A}}
\newcommand{\typeX}{\typefont{X}}
\newcommand{\typeB}{\typefont{B}}
\newcommand{\typeC}{\typefont{C}}
\newcommand{\typeV}{\typefont{V}}
\newcommand{\typeD}{\typefont{D}}
\newcommand{\typeI}{\typefont{I}}
%% RuleName
\newcommand{\rulename}[1]{(\mathrm{#1})}
%% Sequents
\newcommand{\jud}{\vdash}
\newcommand{\vljud}{\rhd}
\newcommand{\cojud}{\vdash_{\co}}
\newcommand{\vl}{\mathtt{v}}
\newcommand{\co}{\mathtt{c}}
% Program font
\newcommand{\prog}[1]{\mathtt{#1}}
\newcommand{\pseq}[3]{#1 \leftarrow #2; #3}
\newcommand{\ppm}[4]{(#1,#2) \leftarrow #3; #4}
\newcommand{\pinl}[1]{\prog{inl}(#1)}
\newcommand{\pinr}[1]{\prog{inr}(#1)}
\newcommand{\pcase}[4]{\prog{ case } #1 \prog{ of } \pinl{#2} \Rightarrow #3 ; \pinr{#2} \Rightarrow #4}
%% Sets of terms
\newcommand{\ValuesBP}[2]{\mathsf{Values}(#1, #2)}
\newcommand{\TermsBP}[2]{\mathsf{Terms}(#1, #2)}
\newcommand{\closValP}[1]{\ValuesBP{\emptyset}{#1}}
\newcommand{\closTermP}[1]{\TermsBP{\emptyset}{#1}}
\newcommand{\closVal}{\closValP{\typeA}}
\newcommand{\closTerm}{\closTermP{\typeA}}
%% Contextual equivalence
\newcommand{\ctxeq}{\equiv_{\prog{ctx}}}
%%%% End of Programming Stuff

%------ Setting lecture info ----------------------------------------------
\newcounter{lectureID}
\stepcounter{lectureID}
\newcommand{\getLecture}{\arabic{lectureID}\xspace}
\newcommand{\setLectureBasic}[1]{
  \title{
    #1
    }
  \author{Jos\'{e} Proen\c{c}a}
  \institute{CISTER -- U.Porto, Porto, Portugal
            \hfill 
            \begin{tabular}{r@{}}
            \url{https://cister-labs.github.io/alg2324}
            \end{tabular}
            }
  \date{Algorithms (CC4010) 2023/2024}
  % logos of institutions
  \titlegraphic{
    \begin{textblock*}{5cm}(4.0cm,6.80cm)
       \includegraphics[scale=0.18]{images/fcup}\hspace*{.85cm}~%
    \end{textblock*}
    \begin{textblock*}{5cm}(8.4cm,7.25cm)
      % \includegraphics[scale=0.50]{images/dcc}
      \includegraphics[scale=0.20]{images/cister}
    \end{textblock*}
  }  
}
\newcommand{\setLecture}[2]{\setcounter{lectureID}{#1}\setLectureBasic{#1. #2}}

%------ Counters for exercises ----------------------------------------------
\newcounter{cExercise}
\newcommand{\exercise}{\stepcounter{cExercise}Ex.\,\arabic{lectureID}.\arabic{cExercise}:\xspace}
\newcommand{\exerciseBack}{\addtocounter{cExercise}{-1}}
\newcommand{\exerciseAdd}{\stepcounter{cExercise}}
\newcommand{\doExercise}[3][0mm]{\begin{exampleblock}{\exercise #2}\wrap{\rule{0pt}{#1}}#3\end{exampleblock}}
\newcommand{\doSimpleExercise}[2][0mm]{\begin{exampleblock}{}\wrap{\rule{0pt}{#1}}\structure{\textbf{\exercise} #2}\end{exampleblock}}

% Slide
\newenvironment{slide}[1]{\begin{frame}\frametitle{#1}}{\end{frame}}

% Misc by José
\newcommand{\wrap}[2][]{\begin{tabular}[#1]{@{}c@{}}#2\end{tabular}}
\newcommand{\mwrap}[1]{\ensuremath{\begin{array}{@{}c@{}}#1\end{array}}}
\def\trans#1{\xrightarrow{#1}}  % - a - > 
\def\Trans#1{\stackrel{#1}{\Longrightarrow}} % =a=> 
\newcommand{\transp}[2][35]{\color{fg!#1}#2}
\newcommand{\transpt}[2][.35]{\tikz{\node[inner sep=1pt,fill opacity=0.5]{#2}}}
\newcommand{\faded}[2][0.4]{{\transparent{#1}#2}} % alternative to "transp" using transparent package
\newcommand{\set}[1]{\left\{ #1 \right\}} % {a,b,...z}
\newcommand{\mi}[1]{\ensuremath{\mathit{#1}}\xspace}
\newcommand{\mf}[1]{\ensuremath{\mathsf{#1}}\xspace}
% \newcommand{\gold}[1]{\textcolor{darkgoldenrod}{#1}\xspace}


%------ using color ---------------------------------------------------------
\definecolor{goldenrod}{rgb}{.80392 .60784 .11373}
\definecolor{darkgoldenrod}{rgb}{.5451 .39608 .03137}
\definecolor{brown}{rgb}{.15 .15 .15}
\definecolor{darkolivegreen}{rgb}{.33333 .41961 .18431}
\definecolor{myGray}{gray}{0.85}
%
%
\newcommand{\red}[1]{\textcolor{red!80!black}{#1}\xspace}
\newcommand{\blue}[1]{\textcolor{blue}{#1}\xspace}
\newcommand{\gold}[1]{\textcolor{darkgoldenrod}{#1}\xspace}
\newcommand{\gray}[1]{\textcolor{myGray}{#1}\xspace}
% \def\alert#1{{\darkgoldenrod #1}}
% \def\alert#1{{\alert{#1}}}
%\def\brw#1{{\brown #1}}
% \def\structure#1{{\blue #1}}
% \def\tstructure#1{\textbf{\darkblue #1}}
%%\def\gre#1{{\green #1}}
\def\gre#1{{\darkolivegreen #1}}
\def\gry#1{{\textcolor{gray}{#1}}}
\def\rdb#1{{\red #1}}
\def\st{\mathbf{.}\,}
\def\laplace#1#2{*\txt{\mbox{ \fcolorbox{black}{myGray}{$\begin{array}{c}\mbox{#1}\\\\#2\\\\\end{array}$} }}}
%\newcommand{\galois}[2]{#1\; \dashv\; #2}



% ----- from LSB
\def\Act{N}
\def\cnil{\mathbf{0}}
\def\cpf#1#2{#1 . #2}                           % a.P
\def\cou#1#2{#1 \mathbin{+} #2}                 % P + Q
%\def\crt#1#2{\mathbin{#1 \setminus_{#2}}}       % P \ A
%\def\crtt#1#2{\mathbin{#1 \setminus\!\setminus_{#2}}}       % P \ A
%\def\crt#1#2{\mathsf{new}\, #2\;  #1}       % P \ A
\def\crt#1#2{#1 \backslash #2}       % P \ A
%\def\crn#1#2{\{#2\}\, #1}                  % P[f]
\def\ainv#1{\overline{#1}}
\def\rtran#1{\stackrel{#1}{\longrightarrow}}
% \def\pair#1{\const{#1}}
\def\pair#1{\langle #1 \rangle}
\def\asor{\mathbin{|}}                    % A | B
\def\setdef#1#2{\mathopen{\{} #1 \asor #2 \mathclose{\}}}
\def\imp{\mathbin{\Rightarrow}}
\def\dimp{\mathbin{\Leftrightarrow}}
\def\rimp{\mathbin{\Leftarrow}}
\def\rra{\longrightarrow}
\def\rcb#1#2#3#4{\def\nothing{}\def\range{#3}\mathopen{\langle}#1 \ #2 \ \ifx\range\nothing::\else: \ #3 :\fi \ #4\mathclose{\rangle}}
\def\aconv#1{#1^{\circ}} 
\def\abv{\stackrel{\rm abv}{=}}


\def\crn#1#2{\mathbin{#1[#2]}}                  % P[f]
\def\couit#1#2{\Sigma_{#1}#2}                  %  + i=1,n
\def\cpar#1#2{#1 \mid #2}                       %  |
\def\ctpar#1#2{#1 \parallel #2}                       %  |
\def\cpars#1#2#3{#1 \mid_{#3} #2}               %  |S

% Spliting frames in 2 columns
\newcommand{\frsplit}[3][.48]{
  \begin{columns}%[T] % align columns
  \begin{column}{#1\textwidth} #2 \end{column} ~~~
  \begin{column}{#1\textwidth} #3 \end{column} \end{columns}
}
\newcommand{\frsplitdiff}[5][]{
  \begin{columns}[#1]%[T] % align columns
  \begin{column}{#2\textwidth} #4 \end{column} ~~~
  \begin{column}{#3\textwidth} #5 \end{column} \end{columns}
}
\newcommand{\frsplitt}[3][.48]{
  \begin{columns}[T] % align columns
  \begin{column}{#1\textwidth} #2 \end{column} ~~~
  \begin{column}{#1\textwidth} #3 \end{column} \end{columns}
}
\newcommand{\col}[2][.48]{\begin{column}{#1\textwidth} #2 \end{column}}
\newcommand{\colb}[3][.48]{\begin{column}{#1\textwidth} \begin{block}{#2} #3 \end{block} \end{column}}

% Spliting frames in 3 columns
\newcommand{\frsplitthree}[4][.31]{
  \begin{columns}%[T] % align columns
  \begin{column}{#1\textwidth} #2 \end{column} ~~~
  \begin{column}{#1\textwidth} #3 \end{column} ~~~
  \begin{column}{#1\textwidth} #4 \end{column} \end{columns}
}
\newcommand{\frsplitdiffthree}[5]{
  \begin{columns}%[T] % align columns
  \begin{column}{#1\textwidth} #3 \end{column} ~~~
  \begin{column}{#1\textwidth} #4 \end{column} ~~~
  \begin{column}{#2\textwidth} #5 \end{column} \end{columns}
}
\newcommand{\frsplittthree}[4][.32]{
  \begin{columns}[T] % align columns
  \begin{column}{#1\textwidth} #2 \end{column} ~
  \begin{column}{#1\textwidth} #3 \end{column} ~
  \begin{column}{#1\textwidth} #4 \end{column} \end{columns}
}


\newcommand{\typerule}[4][]{\ensuremath{\begin{array}[#1]{c}\textsf{\scriptsize ({#2})} \\#3 \\\hline\raisebox{-3pt}{\ensuremath{#4}}\end{array}}}
\newcommand{\styperule}[3][]{\ensuremath{\begin{array}[#1]{c} #2 \\[0.5mm]\hline\raisebox{-4pt}{\ensuremath{#3}}\end{array}}}
\newcommand{\shrk}{\vspace{-3mm}}

\def\caixa#1{\medskip
  \begin{center}
  \fbox{\begin{minipage}{0.9\textwidth}\protect{#1}\end{minipage}}
  \end{center}}

\newcommand{\mybox}[2][0.9]{
  \begin{minipage}{#1\textwidth}\begin{block}{}\centering #2\end{block}\end{minipage}}
\newcommand{\mycbox}[2][0.9]{
  {\\[-5mm]\centering\mybox[#1]{#2}\\[-5mm]}}

%%%%% Tikz
% \usetikzlibrary{arrows.meta, calc, fit, tikzmark}
\usetikzlibrary{%
  positioning
 ,patterns
 ,arrows
 ,arrows.meta
 ,automata
 ,calc
 ,shapes
 ,fit
 ,tikzmark
 ,fadings
 ,decorations.pathreplacing
 ,plotmarks
% ,pgfplots.groupplots
 ,decorations.markings
}
% \tikzset{shorten >=1pt,node distance=2cm,on grid,auto,initial text={},inner sep=2pt}
\tikzstyle{aut}=[shorten >=1pt,node distance=2cm,on grid,auto,initial text={},inner sep=2pt]
\tikzstyle{st}=[circle,draw=black,fill=black!10,inner sep=3pt]
\tikzstyle{sst}=[rectangle,draw=none,fill=none,inner sep=3pt]
\tikzstyle{final}=[accepting]

%%% Uppaal-like diagrams
\newcommand{\uppbox}[3][20mm]{\tikz{
  \node[black!15,fill=black!15,minimum width=#1,align=left](title){\textbf{{\footnotesize #2}}};
  \node[black!15,fill=black!15,left,xshift=4mm]at(title.east){\textbf{{\footnotesize #2}}};
  \node[blue!60!cyan,right] at(title.west){\textbf{{\footnotesize #2}}};
  \node[below,inner sep=2mm,fill=white,xshift=2mm](box)at(title.south){\includegraphics[width=#1]{#3}};
  \node[fit=(title)(box),draw=black,inner sep=0pt]{};
}}

\newcommand{\uppboxv}[3][20mm]{\tikz{
  \node[below,inner sep=2mm,fill=white](box){\includegraphics[height=#1]{#3}};
  \coordinate[yshift=5mm](top)at(box.north);
  \node[fit=(top)(box.north west)(box.north east),inner sep=0pt,fill=black!15](title){};
  \node[blue!60!cyan,right] at(title.west){\textbf{{\footnotesize #2}}};
  \node[fit=(title)(box),draw=black,inner sep=0pt]{};
}}


%% COnfiguring Listings
\lstset{ % basic style
  % language=scala,
  basicstyle=\ttfamily\scriptsize,
  breakatwhitespace=true,
  breaklines=true,
  mathescape,
  % morecomment=[l]{//},
  % morecomment=[n]{/*}{*/},
  % frame=single,                    % adds a frame around the code
  rulecolor=\color{black!40},         % if not set, the frame-color may be changed on line-breaks within not-black text (e.g. comments (green here))
  xleftmargin=1.5mm,
  xrightmargin=1.5mm,
  backgroundcolor=\color{black!5},
  % line numbers
%  numbers=left,  % where to put the line-numbers; possible values are (none, left, right)
 numbersep=5pt, % how far the line-numbers are from the code
 numberstyle=\tiny\color{gray},   
 stepnumber=1,  % the step between two line-numbers. If it is 1 each line will be numbered      
%  xleftmargin=3mm,
%  xrightmargin=1.5mm,
%%%%%
  captionpos=b, % t or b (top or bottom)
  belowcaptionskip=5mm,
%%%%%
  % alsoletter={-},
  % emphstyle=\ttfamily\color{blue}, %\underbar,
  % emphstyle={[2]\ttfamily\color{green!50!black}},
  emphstyle=\bfseries\itshape\color{blue!80!black},       % moreemph={...} - layer keywords
  emphstyle={[2]\itshape\color{red!70!black}},%\underbar} % moreemph={[2]...} - inner keywords
  %
  keywordstyle=\bf\ttfamily\color{red!50!black},
%  commentstyle=\sl\ttfamily\color{gray!70},
  commentstyle=\color{green!60!black},
  stringstyle=\ttfamily\color{purple!60!black},
  morestring=[b]",
  morecomment=[l]{\#},
  frame=single,
  % numberstyle=\tiny, numbers=left, stepnumber=1, firstnumber=1, numberfirstline=true,
  %emph={act,proc,init,sort,map,var,eqn},
  %emph={[2]block,hide,comm,rename,allow,||,<>,sum,&&,=>,true,false},
  % literate=*{->}{{{\color{red!70!black}$\to$}}}{1}
  %            {.}{{{\color{red!70!black}.}}}{1}
  %            {+}{{{\color{red!70!black}\hspace*{1pt}+\hspace*{1pt}}}}{1}
  %            {|}{{{\color{red!70!black}|}}}{1}
  %            {||}{{{\color{red!70!black}|\!\!|}}}{1}
  %            {*}{{{\color{red!70!black}*}}}{1}
  %            {\#}{{{\color{red!70!black}\#}}}{1}
  %            {&}{{{\color{red!70!black}\&}}}{1}
  %            {:=}{{{\color{red!70!black}:=}}}{1}
  %            {=>}{{{\color{red!70!black}=>}}}{1}
  %            % {>}{{{\color{green!65!black}\hspace*{1pt}>\hspace*{1.5pt}}}}{2}
  %            % {<}{{{\color{green!65!black}\hspace*{1.5pt}<\hspace*{1pt}}}}{2}
  %            % {]}{{{\color{green!65!black}\hspace*{1pt}]\hspace*{1.5pt}}}}{1}
  %            % {[}{{{\color{green!65!black}\hspace*{1.5pt}[\hspace*{1pt}}}}{1}
  %            {>}{{{\color{green!65!black}>}}}{1}
  %            {<}{{{\color{green!65!black}<}}}{1}
  %            {]}{{{\color{green!65!black}]}}}{1}
  %            {[}{{{\color{green!65!black}[}}}{1}
  %  morekeywords={Merger1,Fifo2,Lossy3,Init1,Init2}
  % ,emph={act,proc,init,sort,map,var,eqn}
  % ,emph={[2]block,hide,comm,rename,allow,||,<>,sum,&&,=>}
}

\lstdefinestyle{tiny}{basicstyle=\ttfamily\relsize{-7}}

% Include slides from others
\newcommand{\byothers}[3]{{
\begin{frame}{}~\mycbox{\Large slides by #1\\pages #2}\end{frame}{}
\setbeamercolor{background canvas}{bg=}
\includepdf[pages=#2]{../../others/#3}
}}

%-------------- template --------------------------------------------------
\usetheme{metropolis}
\usepackage{appendixnumberbeamer}

% Base colors (from metropolis theme)
\definecolor{metDarkBrown}{HTML}{604c38}
\definecolor{metDarkTeal}{HTML}{23373b}
\definecolor{metLightBrown}{HTML}{EB811B}
\definecolor{metLightGreen}{HTML}{14B03D}

 

\metroset{numbering=fraction,progressbar=frametitle}

% \setbeamercolor*{structure}{fg=blue!80!black}
\setbeamercolor*{structure}{fg=metLightGreen}

% % \definecolor{MainColour}{rgb}{0., 0.25, 0.8}
% \colorlet{MainColour}{blue!50!black}
% \colorlet{BgColour}{blue!10}
% \colorlet{BarColour}{blue!50!black}


% %\usetheme{CambridgeUS}%{Copenhagen}%{Frankfurt}%{Singapore}%{CambridgeUS}
% \usecolortheme[named=MainColour]{structure} 
% \useoutertheme[subsection=false]{miniframes}
% \useinnertheme{circles}
% %\useinnertheme[shadow=false]{rounded}
% \setbeamertemplate{blocks}[rounded][shadow=false]

% \setbeamercovered{transparent} 
% \setbeamertemplate{navigation symbols}{} %Remove navigation bar
% \setbeamertemplate{footline}[frame number] % add page number
% \setbeamercolor{postit}{fg=MainColour,bg=BgColour}
% \setbeamercolor{structure}{bg=black!10}
% %\setbeamercolor{palette primary}{use=structure,fg=red,bg=green}
% %\setbeamercolor{palette secondary}{use=structure,fg=red!75!black,bg=green}
% \setbeamercolor{palette tertiary}{use=structure,bg=BarColour,fg=white}
% %\setbeamercolor{palette quaternary}{fg=black,bg=green}
% %\setbeamercolor{normal text}{fg=black,bg=white}
% %\setbeamercolor{block title alerted}{fg=red,bg=green}
% %\setbeamercolor{block title example}{bg=black!10,fg=green}
\setbeamercolor{block body}{bg=black!5}

% \setbeamercolor{block title alerted}{bg=red!25}
% \setbeamercolor{block body alerted}{bg=red!10}

% \setbeamercolor{block title example}{bg={rgb:green,2;black,1;white,5}}
\setbeamercolor{block body example}{bg={rgb:green,2;black,1;white,20}}
\setbeamercolor{block body alerted}{bg={metLightBrown!25}}

% \setbeamertemplate{itemize item}{\color{black!10}$\blacksquare$}
\setbeamercolor{itemize item}{fg=metDarkTeal}
\setbeamercolor{itemize subitem}{fg=metDarkTeal}

\setbeamercolor{graybc}{fg=black,bg=black!10}
\newcommand{\myblock}[1]{\begin{beamercolorbox}[dp=1ex,center,rounded=true]%
  {graybc} {\large \textbf{#1}} \end{beamercolorbox}}%


% Configuring the foot line
\setbeamertemplate{footline}
{
  \leavevmode%
  \hbox{%
  \begin{beamercolorbox}[wd=.4\paperwidth,ht=2.25ex,dp=1ex,center]{author in head/foot}%
    \usebeamerfont{author in head/foot}\insertshortauthor
  \end{beamercolorbox}%
  \begin{beamercolorbox}[wd=.5\paperwidth,ht=2.25ex,dp=1ex,center]{title in head/foot}%
    \usebeamerfont{title in head/foot}\insertsection
  \end{beamercolorbox}%
  \begin{beamercolorbox}[wd=.1\paperwidth,ht=2.25ex,dp=1ex,right]{date in head/foot}%
    \insertframenumber{} / \inserttotalframenumber\hspace*{2ex} 
  \end{beamercolorbox}}%
  \vskip0pt%
}
% No configuration symbols
\setbeamertemplate{navigation symbols}{}


%----------------------------------------------------------------------------

\begin{document}

\setLecture{6}{Data Structures [WiP]}
\frame[plain]{\titlepage}


\begin{frame}[t]\frametitle{Overview}

  \begin{itemize}
    \item Stacks/Queues/PriorityQueues  %(\alert{minHeap})
    \item Hashtables/Search trees
    \item Graphs
    \begin{itemize}
      \item Depth/Breathfirst traversals
      \item Acyclic -- topological order
      \item Transitive closure
      \item Minimum spanning tree
      \item Shortest/longest path
    \end{itemize}
  \end{itemize}

  % transversal: dynamic programming?, greedy programming?

\end{frame}


\begin{frame}\frametitle{Motivation}
  \centering

  \begin{block}{We have seen that}
    Different \alert{data structures} are better at different \structure{operations}
  \end{block}

  \begin{block}{We will see}
    Useful data structures and associated operations (code)
  \end{block}

  \begin{exampleblock}{Examples}
    Arrays can have operations to implement sets, multisets, trees, etc.
  \end{exampleblock}

\end{frame}


\section{Sets and Sequences}

\begin{frame}[fragile]\frametitle{Sets and Multisets}
  
\begin{columns}
\begin{column}{.48\textwidth}
\begin{lstlisting}[language=C++,emph={SetInt,MSetInt}]
#define MAXS 100
typedef char SetInt [MAXS] ;
\end{lstlisting}

% \only<1>{
Given \texttt{SetInt s}:
\\$5 \in \texttt{s} ~\Leftrightarrow~ \texttt{s[5]!=0}$
% }
% \only<2->{
% \begin{lstlisting}[language=C++]
% void initSet (SetInt);
% int searchSet (SetInt , int);
% int addSet (SetInt, int);
% int emptySet (SetInt);
% void unionSet (SetInt , SetInt , SetInt );
% void intersectSet (SetInt , SetInt , SetInt );
% void differenceSet (SetInt , SetInt , SetInt );
% \end{lstlisting}
% }
\end{column}
\begin{column}{.48\textwidth}
\begin{lstlisting}[language=C++,emph={SetInt,MSetInt}]
#define MAXMS 100
typedef int MSetInt [MAXS] ;
\end{lstlisting}

% \only<1>{
Given \texttt{MSetInt ms}:
\\$\{4,4\} \subseteq \texttt{ms} ~\Leftrightarrow~ \texttt{ms[4]}\leq 2$
% }
% \only<2->{
% \begin{lstlisting}[language=C++]
% void initMSet ( MSetInt ) ; 
% int searchMSet ( MSetInt , int ) ;
% int addMSet ( MSetInt , int ) ;
% int emptyMSet ( MSetInt ) ;
% void unionMSet (MSetInt , MSetInt , SetInt );
% void intersectMSet (MSetInt , MSetInt , MSetInt ); 
% void differenceMSet (MSetInt , MSetInt , MSetInt );
% \end{lstlisting}
% }
\end{column}
\end{columns}

\end{frame}


\begin{frame}[fragile]\frametitle{Sets and Multisets -- operations}
\centering

\begin{columns}
\begin{column}{.48\textwidth}
\begin{lstlisting}[language=C++,emph={MSetInt,SetInt}]
void initSet      (SetInt);
int  searchSet    (SetInt, int);
int  addSet       (SetInt, int);
int  emptySet     (SetInt);
void unionSet     (SetInt, SetInt,
                   SetInt);
void intersectSet (SetInt, SetInt,
                   SetInt);
void differenceSet(SetInt, SetInt,
                   SetInt);
\end{lstlisting}
%
\end{column}
\begin{column}{.50\textwidth}
%
\begin{lstlisting}[language=C++,emph={MSetInt,SetInt}]
void initMSet      (MSetInt); 
int  searchMSet    (MSetInt, int);
int  addMSet       (MSetInt, int);
int  emptyMSet     (MSetInt);
void unionMSet     (MSetInt, MSetInt,
                    SetInt);
void intersectMSet (MSetInt, MSetInt,
                    MSetInt); 
void differenceMSet(MSetInt, MSetInt,
                    MSetInt);
\end{lstlisting}
%
\end{column}
\end{columns}

\doSimpleExercise{What is the expected cost of each function? Could you implement them?}
\end{frame}

\begin{frame}[fragile]\frametitle{Sequences -- Recall linked lists}
  
\begin{lstlisting}[language=C++,emph={list,LInt}]
typedef struct list { int value ;
struct list *next;
} *LInt;
\end{lstlisting}

  % add, remove, append, concat
\begin{columns}
\begin{column}{.48\textwidth}
\begin{lstlisting}[language=C++, emph={prev,LInt}]
LInt add (int x, LInt l) {
  LInt new =
    malloc(sizeof(struct list));
  if (new != NULL) {
    new->value=x;
    new->next=l ;
  }
return new;
}
\end{lstlisting}
%
\end{column}
\begin{column}{.50\textwidth}
%
\begin{lstlisting}[language=C++, emph={prev,LInt}]
LInt dda (int x, LInt l) {
  LInt pt = l;
  while (pt != NULL) pt = pt->next;
  pt = malloc(sizeof(struct list));
  pt -> next = x;
  pt -> next = NULL ;
  return l ;
}
\end{lstlisting}
%
\end{column}
\end{columns}
\end{frame}



\begin{frame}[fragile]\frametitle{Sequences -- Recall linked lists (fixed)}
  
\begin{lstlisting}[language=C++,emph={list,LInt}]
typedef struct list { int value ;
struct list *next;
} *LInt;
\end{lstlisting}

  % add, remove, append, concat
\begin{columns}
\begin{column}{.48\textwidth}
\begin{lstlisting}[language=C++, emph={prev,LInt}]
LInt add (int x, LInt l) {
  LInt new = malloc(sizeof(struct list ));
  if (new != NULL) {
    new->value=x;
    new->next=l ;
  }
return new;
}
\end{lstlisting}
%
\end{column}
\begin{column}{.50\textwidth}
%
\begin{lstlisting}[language=C++, emph={prev,LInt}]
LInt dda (int x, LInt l) {
  LInt pt = l, prev;
  while (pt != NULL) {
    prev = pt; pt = pt->next; }
  pt = malloc(sizeof(struct list));
  pt->next = x;
  pt->next = NULL ;
  if (l==NULL)  l = pt;
  else prev->prox = pt;
  return l;
}
\end{lstlisting}
%
\end{column}
\end{columns}

~\\[-6mm]
\doSimpleExercise{What is the possible complexity of 
  \texttt{lookup, concat, reverse}?
}
\end{frame}

\begin{frame}[fragile]\frametitle{Sequences -- reverse analysis}
  
\begin{columns}
\begin{column}{.48\textwidth}
\begin{lstlisting}[language=C++, emph={reverse1,reverse2,LInt}]
LInt reverse1 (LInt l) {
  LInt r, pt;
  if (l==NULL || l->next==NULL) r=l;
  else {
    r = pt = reverse1 (l->next);
    while (pt->next != NULL)
      pt = pt->next;
    pt->next = l;
    l->next = NULL;
  }
  return r; }
\end{lstlisting}
%
\end{column}
\begin{column}{.50\textwidth}
%
\begin{lstlisting}[language=C++, emph={reverse1,reverse2,LInt}]
LInt reverse2 (LInt l) {
  LInt r, tmp;
  r = NULL;
  while (l !=NULL) {
    tmp=l; l=l->next;
    tmp->next=r; r=tmp;
  }
  return r;
}
\end{lstlisting}
%
\end{column}
\end{columns}

~\\[-6mm]
\doSimpleExercise{What is the complexity of each \texttt{reverse}?}
~\\[-12mm]
\doSimpleExercise{What is the (informal) loop invariant in \texttt{reverse2}, assuming:
\\~~~ \texttt{pre:l==l$_0$} and \texttt{post:r==rev(l$_0$)}?}
% r==l0?  rev(l)++r == rev(l0)
\end{frame}


\begin{frame}\frametitle{Complexity of collections in Scala}
    
    \centering

    \begin{block}{}
    {\huge \url{https://docs.scala-lang.org/overviews/collections-2.13/performance-characteristics.html}}      
    \end{block}


\end{frame}


\section{Buffers (stacks and queueus)}


\begin{frame}[fragile]\frametitle{Stacks}

\begin{columns}
\begin{column}{.32\textwidth}
%
\begin{lstlisting}[language=C++, emph={stack,Stack}]
#define MAX 1000
typedef struct stack {
  int values [MAX];
  int sp;
} Stack;
\end{lstlisting}
\only<2->{with static arrays}
%
\end{column}
\begin{column}{.32\textwidth}
%
\begin{lstlisting}[language=C++, emph={stack,Stack}]
typedef struct cell {
  int value;
  struct cell *next;
} Cell , *Stack;
$~$
\end{lstlisting}
\only<2->{with linked lists}
%
\end{column}
\begin{column}{.32\textwidth}
%
\begin{lstlisting}[language=C++, emph={stack,Stack}]
typedef struct stack {
  int size;
  int *values;
  int sp;
} Stack;
\end{lstlisting}
\only<2->{with dynamic arrays}
%
\end{column}
\end{columns}
  

\only<2->{\doSimpleExercise{(Informally) what is the complexity of: \texttt{push}, \texttt{pop}, \texttt{head}?}}
  % with static arrays

  % with linked lists

  % with dynamic arrays (initStack, push/double, pop)

\end{frame}

\begin{frame}[fragile]\frametitle{Exercise: Push-pop with dynamic arrays}
    

\begin{columns}
\begin{column}{.48\textwidth}
\begin{lstlisting}[language=C++, emph={push,pop,doubleArray,realloc}]
void push (Stack *s , int x){
  if (s->sp == s->size)
    doubleArray (s);
  if (r == 0)
    s->values[s->sp++] = x;
}

void doubleArray (Stack *s){
  s->size *= 2;
  s->values =
    realloc(s->values, s->size);
}
\end{lstlisting}
%
\end{column}
\begin{column}{.50\textwidth}
%
\begin{lstlisting}[language=C++, emph={push,pop,doubleArray,realloc}]
int pop (Stack *s){
  // reduces by half when only
  // 25% capacity is used
  ...
}

void halfArray (Stack *s){
  ...
}
\end{lstlisting}
%
\end{column}
\end{columns}

\doSimpleExercise{Implement the optimised \texttt{pop} function and discuss its complexity.}

\end{frame}



\begin{frame}[fragile]\frametitle{Queues}
    
\begin{columns}
\begin{column}{.29\textwidth}
%
\begin{lstlisting}[language=C++, emph={queue,Queue}]
#define MAX 1000
typedef struct queue
{
  int values [MAX];
  int start, size;
} Queue;
$~$
$~$
\end{lstlisting}
\only<2->{with static arrays (circular)}
%
\end{column}
\begin{column}{.37\textwidth}
%
\begin{lstlisting}[language=C++, emph={cell,Cell,Queue}]
typedef struct cell {
  int value ;
  struct cell *prox ;
} Cell ;

typedef struct queue {
  struct cell *start, *end;
} Queue;
\end{lstlisting}
\only<2->{with linked lists\\~}
%
\end{column}
\begin{column}{.29\textwidth}
%
\begin{lstlisting}[language=C++, emph={queue,Queue}]
typedef struct queue
{
  int max;
  int *values;
  int start, size;
} Queue;
$~$
$~$
\end{lstlisting}
\only<2->{with dynamic arrays (circular)}
%
\end{column}
\end{columns}

\only<2->{\doSimpleExercise{(Informally) what is the complexity of: \texttt{init}, \texttt{isEmpty}, \texttt{enqueue}, \texttt{dequeue}?}}

\end{frame}



% \begin{frame}\frametitle{Queues}
    
%   with static arrays (circular)

%   with linked lists (store 2 pointers)
  
% \end{frame}


\begin{frame}[fragile]\frametitle{Priority Queues}
\begin{columns}
\begin{column}{.48\textwidth}
%
\begin{itemize}
  \item Binary tree
  \item Each node is larger than any of its children
  \item Implemented as an array
\end{itemize}
%
\end{column}
\begin{column}{.50\textwidth}
%
\begin{lstlisting}[language=C++,emph={prQueue}]
#define MAX 1000
typedef struct prQueue {
  int values [MAX];
  int size ;
} PriorityQ ;
\end{lstlisting}
%
\end{column}
\end{columns}


\bigskip

  \begin{exampleblock}{Tree example in the board}
    \centering
    \texttt{size=17~}%
    \textcolor{black!40}{%
    \texttt{~~0~~1~~2~~3~~4~~5~~6~~7~~8~~9 10 11 12 13 14 15 16~}}
    \\\texttt{values:~~}%
    \texttt{[10 15 11 16 22 35 20 21 23 34 37 80 43 22 25 24 28]}
  \end{exampleblock}
  % 10
  % 15 11
  % 16 22 35 20...

  % exercises (bubble up, bubble down, complexity)

\end{frame}


\begin{frame}\frametitle{Exercises}
  
  \doExercise{Using the previous example, provide an expression to:}{
    \noindent
    \!\!1. calculate the index of the \emph{left} tree given a position \texttt{i} % 2i+1
    \\2. calculate the index of the \emph{right} tree given a position \texttt{i} % 2i+2
    \\3. calculate the index of the \emph{parent} of a given a position \texttt{i} % (i-1)/2
    \\4. calculate the index of the index of the \emph{first leaf}
  }

  \doExercise{Define \texttt{bubbleUp(int i, int h[])}}{
    Fixes a min-heap by swapping the \texttt{i}-th element with the parent while needed.
  }

  \doExercise{Define \texttt{bubbleDown(int i, int h[], int N)}}{
    Fixes a min-heap by swapping the \texttt{i}-th element with one of the children while needed.
  }
\end{frame}


\begin{frame}\frametitle{Exercises}
  \doExercise{Define the following operations:}{
    \noindent
    \!\!- \texttt{void empty (PriorityQueue *q)} -- initialises the queue;
    \\- \texttt{int isEmpty (PriorityQueue *q)} -- tests if \texttt{q} is empty;
    \\- \texttt{int add (int x, PriorityQueue *q)} -- adds a value \texttt{x}, returning 0 when the queue is full;
    \\- \texttt{int remove (PriorityQueue *q, int *rem)} -- removes the next element, and copies it to $rem$.
  }
\end{frame}


\section{Dictionaries}

\begin{frame}\frametitle{Hashtables}

\alert{Dictionary:} maps \structure{keys} to \structure{values}
\\
(Keys are unique)

\begin{block}{Idea}
  - \alert{\emph{Magic} function \texttt{hash}} converts a key into an \structure{index} (number).
  \\- This \structure{index} points to the position of an array where the value \emph{should} be found.
  \\- Usually the size of the array is \alert{less} than the set of possible keys, i.e., \structure{\texttt{hash} is not injective}.
  \\- If 2 keys have the same \texttt{hash} value, there is a \alert{colision} that must be mitigated (alternative solutions exist).
\end{block}

\end{frame}


\begin{frame}\frametitle{Hashtables: Closed and Open Addressing}

\begin{columns}
  \begin{column}{0.48\textwidth}
    \begin{alertblock}{Closed Addressing (or chaining)}
      - Table = \emph{array of linked lists}
      \\
      - Find value of key \alert{\texttt{k}}:
      \\~~~- go to index \texttt{hash(k)}
      \\~~~- traverse list until \texttt{k}
    \end{alertblock}
  \end{column}
  \begin{column}{0.48\textwidth}
    \begin{exampleblock}{Open Addressing}
      - Table = \emph{just an array}
      \\
      - Find value of key \alert{\texttt{k}}:
      \\~~~- go to index \texttt{hash(k)}
      \\~~~- \emph{``jump''} until \texttt{k}
    \end{exampleblock}
  \end{column}
\end{columns}

\bigskip

\begin{block}{Some concerns}
  - Use dynamic arrays (grow when the \alert{load factor} (\#keys/HSIZE) gets high)
  \\~~~-Need to \emph{rehash}  
  \\- Smart \emph{jumps} (\texttt{probe} function to know where to jump)
  \\- Need to \emph{garbage collect} in open addressing
\end{block}

\end{frame}


\begin{frame}\frametitle{Intuition: Hashtables with Closed Addressing}
  \centering
  \includegraphics[width=0.6\textwidth]{images/HT-closed.png}
  \\
  {\footnotesize \textcolor{gray}{(from Wikipedia)}}
\end{frame}

\begin{frame}[fragile]\frametitle{Hashtables with Closed Addressing}
\begin{columns}
\begin{column}{.61\textwidth}
%
\begin{itemize}
  \item \texttt{int hash(\alert{\texttt{int}} k, int size);}
  \item \texttt{void initTab(HTChain h);}
  \item \texttt{int lookup(HTChain h, \alert{\texttt{int}} k, int *i);}
  \item \texttt{int update(HTChain h, \alert{\texttt{int}} k, int i);}
  \item \texttt{int remove(HTChain h, \alert{\texttt{int}} k);}
\end{itemize}
%
\end{column}
\begin{column}{.34\textwidth}
%
\begin{lstlisting}[language=C++,emph={HTChain,Bucket}]
#define HSIZE 1000

typedef struct bucket {
  $\alert{\texttt{int}}$ key;
  int info;
  struct bucket *next;
} *Bucket;

typedef Bucket
   HTChain[HSIZE];
\end{lstlisting}
%
\end{column}
\end{columns}

\doSimpleExercise{Implement \texttt{lookup}}
  % int p = hash (k,HSIZE); Bucket it;
  % for (it = h[p]; it!=NULL && it->key!=k; it=it->next);
  % if (it!=NULL) { *i = it->info; return 1;}
  % else return 0;
~\\[-10mm]
\doSimpleExercise{(Informally) what is the expected complexity of each function?}
\end{frame}


\begin{frame}\frametitle{Intuition: Hashtables with Open Addressing}
  \centering
  ~\\[-12mm]
  \includegraphics[width=0.6\textwidth]{images/HT-open.png}
  \\
  {\footnotesize \textcolor{gray}{(from Wikipedia)}}
\end{frame}


\begin{frame}[fragile]\frametitle{Hashtables with Open Addressing}
\begin{columns}
\begin{column}{.71\textwidth}
%
\begin{itemize}
  \item \texttt{int hash(\alert{\texttt{int}} k, int size);}
  \item \texttt{void initTab(HashTable h);}
  \item \texttt{void lookup(HashTable h, \alert{\texttt{int}} k, int *i);}
  \item \texttt{void update(HashTable h, \alert{\texttt{int}} k, int i);}
  \item \textcolor{black!40}{
          \texttt{void remove(HashTable h, \alert{\texttt{int}} k);}}
  \item \structure{
          \texttt{int find\_probe (HashTable h, \alert{\texttt{int}} k)}}
          \\~~~- linear vs. quadratic probing (why quadratic?)
\end{itemize}
%
\end{column}
\begin{column}{.34\textwidth}
%
\begin{lstlisting}[language=C++,emph={HashTable,Bucket}]
#define HSIZE 1000
#define STATUSFREE 0
#define STATUSUSED 1

typedef struct bucket {
  int status ;
  $\alert{\texttt{int}}$ key;
  int info;
} Bucket ;

typedef Bucket
  HashTable [HSIZE];
\end{lstlisting}
%
\end{column}
\end{columns}

~\\[-7mm]
\doSimpleExercise{Define a linear probing function and \texttt{update}.}
% ~\\[-10mm]
% \doSimpleExercise{(Informally) what is the expected complexity of each function?}
\end{frame}



\begin{frame}\frametitle{Lookups: Open vs. Closed}
  \centering
  \includegraphics[width=0.6\textwidth]{images/HT-lookups.png}
  \\
  {\footnotesize \textcolor{gray}{(from Wikipedia)}}
\end{frame}
    




\begin{frame}[fragile]\frametitle{Removing with Open Addressing}
\begin{columns}
\begin{column}{.71\textwidth}
%
\begin{itemize}
  \item \textcolor{black!40}{\texttt{int hash(\alert{\texttt{int}} k, int size);}}
  \item \textcolor{black!40}{\texttt{void initTab(HashTable h);}}
  \item \textcolor{black!40}{\texttt{void lookup(HashTable h, \alert{\texttt{int}} k, int *i);}}
  \item \textcolor{black!40}{\texttt{void update(HashTable h, \alert{\texttt{int}} k, int i);}}
  \item \textcolor{black!40}{\texttt{int find\_probe (HashTable h, \alert{\texttt{int}} k);}}
  \item \texttt{void \structure{remove}(HashTable h, \alert{\texttt{int}} k);}
\end{itemize}
%%% find_probe gets more complex.
%
\end{column}
\begin{column}{.34\textwidth}
%
\begin{lstlisting}[language=C++,emph={HashTable,Bucket}]
#define HSIZE 1000
#define STATUSFREE 0
#define STATUSUSED 1
#define $\structure{\texttt{STATUSDEL 2}}$

typedef struct bucket {
  int status ;
  $\alert{\texttt{int}}$ key;
  int info;
} Bucket ;

typedef Bucket
  HashTable [HSIZE];
\end{lstlisting}
%
\end{column}
\end{columns}

~\\[-6mm]
\doSimpleExercise{How would you implement \texttt{update}?
% ~\\[-12mm]
% \doSimpleExercise{
\\How would you implement a \emph{garbageCollect} that removes deleted cells? 
\\What is their complexity?}
\end{frame}

% \section{Balanced trees}

\begin{frame}\frametitle{More dictionaries: balanced trees}

\begin{block}{We will see:}
  - Height- and weight-balanced tree
  \\- Self-balancing binary search tree
  \\~~~- AVL tree
  % Adelson-Velsky and Landis
  % oldest self-balancing binary search tree data structure to be invented.
% \begin{tabular}{lcc}
% \toprule
% Function &  Amortized & Worst Case
% \\\midrule
% Search     & ${\Theta(\log n)}$ & ${\mathcal{O}(\log n)}$\\
% Insert     & ${\Theta(\log n)}$ & ${\mathcal{O}(\log n)}$\\
% Delete     & ${\Theta(\log n)}$ & ${\mathcal{O}(\log n)}$\\
% \bottomrule
% \end{tabular}
  \\~~~- Red-black tree
  %  It requires in the worst case a small number, O(log n) in Big O notation, where nn is the number of objects in the tree, on average or amortized O(1), a constant number, of color changes (which are very quick in practice); and no more than three tree rotations (two for insertion).
  % As of Java 8, the HashMap has been modified such that instead of using a LinkedList to store different elements with colliding hashcodes, a red–black tree is used.
% \toprule
% Function &  Amortized & Worst Case
% \\\midrule
% Search     & ${\mathcal{O}(\log n)}$ & ${\mathcal{O}(\log n)}$\\
% Insert     & ${\mathcal{O}(1)}$ & ${\mathcal{O}(\log n)}$\\
% Delete     & ${\mathcal{O}(1)}$ & ${\mathcal{O}(\log n)}$\\
% \bottomrule
% \end{tabular}
\end{block}

% [Binary Search Trees / AVL (self) / B-Tree / Red-Black]

\end{frame}


\begin{frame}\frametitle{Binary Balanced Search Trees}

  \begin{columns}[t]
  \begin{column}{0.48\textwidth}
    \begin{block}{Height-balanced}
      - more used
      \\- left-height = right-height ± 1 (AVL)
      \\- height = $\log n$
    \end{block}      
  \end{column}    
  \begin{column}{0.48\textwidth}
    \begin{block}{Weight-balanced}
      - less used
      \\- $\text{leafs-left/right} \geq \alpha \times \text{leafs}$, $0<\alpha<1$
    \end{block}      
  \end{column}    
  \end{columns}    


\end{frame}

\section{Graphs}

\begin{frame}\frametitle{Overview}
    
  \begin{itemize}
    \item Depth/Breathfirst traversals
    \item Acyclic -- topological order
    \item Transitive closure
    \item Minimum spanning tree
    \item Shortest/longest path
  \end{itemize}

\end{frame}

\end{document}
